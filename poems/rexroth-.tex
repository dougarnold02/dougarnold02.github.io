\begin{poem}
{Untitled}
{Yamabe no Akahito, trans. K. Rexroth}
 The mists rise over 
 The waters at Asuka;
 Memory does not
 Pass away so easily.\\

Asuka gawa
Kawa yodo sarazu
Tatsu kiri no
Omoi sugu beki
Koi ni aranuku ni
\end{poem}
      
\begin{poem}
    {Autumn in California}
    {Kenneth Rexroth}

    Autumn in California is a mild 
    And anonymous season, hills and valleys 
    Are colorless then, only the sooty green 
    Eucalyptus, the conifers and oaks sink deep 
    Into the haze; the fields are plowed, bare, waiting; 
    The steep pastures are tracked deep by the cattle; 
    There are no flowers, the herbage is brittle. 
    All night along the coast and the mountain crests 
    Birds go by, murmurous, high in the warm air. 
    Only in the mountain meadows the aspens 
    Glitter like goldfish moving up swift water; 
    Only in the desert villages the leaves 
    Of the cottonwoods descend in smoky air. 
    \hfill       Once more I wander in the warm evening 
    Calling the heart to order and the stiff brain 
    To passion. I should be thinking of dreaming, loving, dying, 
    Beauty wasting through time like draining blood, 
    And me alone in all the world with pictures 
    Of pretty women and the constellations. 
    But I hear the clocks in Barcelona strike at dawn 
    And the whistles blowing for noon in Nanking. 
    I hear the drone, the snapping high in the air 
    Of planes fighting, the deep reverberant 
    Grunts of bombardment, the hasty clamor 
    Of anti-aircraft. 
    \hfill                    In Nanking at the first bomb, 
    A moon-faced, willowy young girl runs into the street, 
    Leaves her rice bowl spilled and her children crying, 
    And stands stiff, cursing quietly, her face raised to the sky. 
    Suddenly she bursts like a bag of water, 
    And then as the blossom of smoke and dust diffuses, 
    The walls topple slowly over her. 
    \hfill                                       I hear the voices 
    Young, fatigued and excited, of two comrades 
    In a closed room in Madrid. They have been up 
    All night, talking of trout in the Pyrenees, 
    Spinoza, old nights full of riot and sherry, 
    Women they might have had or almost had, 
    Picasso, Velasquez, relativity. 
    The candlelight reddens, blue bars appear 
    In the cracks of the shutters, the bombardment 
    Begins again as though it had never stopped, 
    The morning wind is cold and dusty, 
    Their furloughs are over. They are shock troopers, 
    They may not meet again. The dead light holds 
    In impersonal focus the patched uniforms, 
    The dog-eared copy of Lenin's Imperialism, 
    The heavy cartridge belt, holster and black revolver butt. 
    \hfill        The moon rises late over Mt. Diablo, 
    Huge, gibbous, warm; the wind goes out, 
    Brown fog spreads over the bay from the marshes, 
    And overhead the cry of birds is suddenly 
    Loud, wiry, and tremulous. 

\hfill        [1938]
\end{poem}


\begin{poem}
    {August 22, 1939}
    {Kenneth Rexroth}

    \textit{``. . . when you want to distract your mother from the
      discouraging soulness, I will tell you what I used to do. To
      take her for a long walk in the quiet country, gathering
      wildflowers here and there, resting under the shade of trees,
      between the harmony of the vivid stream and the tranquillity of
      the mother-nature, and I am sure she will enjoy this very much,
      as you surely will be happy for it. But remember always, Dante,
      in the play of happiness, don't use all for yourself only, but
      down yourself just one step, at your side and help the weak
      ones that cry for help, help the prosecuted and the victim;
      because they are your friends; they are the comrades that fight
      and fall as your father and Bartolo fought and fell yesterday,
      for the conquest of the joy of freedom for all and the poor
      workers. In this struggle of life you will find more love and
      you will be loved.'' \\ 

      --- Nicola Sacco to his son Dante, Aug. 18, 1927.}\\

     \textit{Angst und Gestalt und Gebet ---Rilke}\\ 

    What is it all for, this poetry, 
    This bundle of accomplishment 
    Put together with so much pain? 
    Twenty years at hard labor, 
    Lessons learned from Li Po and Dante, 
    Indian chants and gestalt psychology; 
    What words can it spell, 
    This alphabet of one sensibility? 
    The pure pattern of the stars in orderly progression, 
    The thin air of fourteen-thousand-foot summits, 
    Their Pisgah views into what secrets of the personality, 
    The fire of poppies in eroded fields, 
    The sleep of lynxes in the noonday forest, 
    The curious anastomosis of the webs of thought, 
    Life streaming ungovernably away, 
    And the deep hope of man. 
    The centuries have changed little in this art, 
    The subjects are still the same. 
    ``For Christ's sake take off your clothes and get into bed, 
    We are not going to live forever.'' 
    ``Petals fall from the rose,'' 
    We fall from life, 
    Values fall from history like men from shellfire, 
    Only a minimum survives, 
    Only an unknown achievement. 
    They can put it all on the headstones, 
    In all the battlefields, 
    ``Poor guy, he never knew what it was all about.'' 
    Spectacled men will come with shovels in a thousand years, 
    Give lectures in universities on cultural advances, cultural lags. 
    A little more garlic in the soup, 
    A half-hour more in bed in the morning, 
    Some of them got it, some of them didn't; 
    The things they dropped in their hurry 
    Are behind the glass cases of dusky museums. 
    This year we made four major ascents, 
    Camped for two weeks at timberline, 
    Watched Mars swim close to the earth, 
    Watched the black aurora of war 
    Spread over the sky of a decayed civilization. 
    These are the last terrible years of authority. 
    The disease has reached its crisis, 
    Ten thousand years of power, 
    The struggle of two laws, 
    The rule of iron and spilled blood, 
    The abiding solidarity of living blood and brain. 
    They are trapped, beleaguered, murderous, 
    If they line their cellars with cork 
    It is not to still the pistol shots, 
    It is to insulate the last words of the condemned. 
    ``Liberty is the mother 
    Not the daughter of order.'' 
    ``Not the government of men 
    But the administration of things.'' 
    ``From each according to his ability, 
    Unto each according to his needs.'' 
    We could still hear them, 
    Cutting steps in the blue ice of hanging glaciers, 
    Teetering along shattered ar\^etes. 
    The cold and cruel apathy of mountains 
    Has been subdued with a few strands of rope 
    And some flimsy iceaxes, 
    There are only a few peaks left. 
    Twenty-five years have gone since my first sweetheart. 
    Back from the mountains there is a letter waiting for me. 
    ``I read your poem in the New Republic. 
    Do you remember the undertaker's on the corner, 
    How we peeped in the basement window at a sheeted figure 
    And ran away screaming? Do you remember? 
    There is a filling station on the corner, 
    A parking lot where your house used to be, 
    Only ours and two other houses are left. 
    We stick it out in the noise and carbon monoxide.'' 
    It was a poem of homesickness and exile, 
    Twenty-five years wandering around 
    In a world of noise and poison. 
    She stuck it out, I never went back, 
    But there are domestic as well as imported 
    Explosions and poison gases. 
    Dante was homesick, the Chinese made an art of it, 
    So was Ovid and many others, 
    Pound and Eliot amongst them, 
    Kropotkin dying of hunger, 
    Berkman by his own hand, 
    Fanny Baron biting her executioners, 
    Mahkno in the odor of calumny, 
    Trotsky, too, I suppose, passionately, after his fashion. 
    Do you remember? 
    What is it all for, this poetry, 
    This bundle of accomplishment 
    Put together with so much pain? 
    Do you remember the corpse in the basement? 
    What are we doing at the turn of our years, 
    Writers and readers of the liberal weeklies? 

\hfill [1939]
\end{poem}


\begin{poem}
    {From the Paris Commune to
    the Kronstadt Rebellion}
    {Kenneth Rexroth} 

    Remember now there were others before this; 
    Now when the unwanted hours rise up, 
    And the sun rises red in unknown quarters, 
    And the constellations change places, 
    And cloudless thunder erases the furrows, 
    And moonlight stains and the stars grow hot. 
    Though the air is fetid, conscripted fathers, 
    With the black bloat of your dead faces; 
    Though men wander idling out of factories 
    Where turbine and hand are both freezing; 
    And the air clears at last above the chimneys; 
    Though mattresses curtain the windows; 
    And every hour hears the snarl of explosion; 
    Yet one shall rise up alone saying: 
    ``I am one out of many, I have heard 
    Voices high in the air crying out commands; 
    Seen men's bodies burst into torches; 
    Seen faun and maiden die in the night air raids; 
    Heard the watchwords exchanged in the alleys; 
    Felt hate speed the blood stream and fear curl the nerves. 
    I know too the last heavy maggot; 
    And know the trapped vertigo of impotence. 
    I have traveled prone and unwilling 
    In the dense processions through the shaken streets. 
    Shall we hang thus by taut navel strings 
    To this corrupt placenta till we're flyblown; 
    Till our skulls are cracked by crow and kite 
    And our members become the business of ants, 
    Our teeth the collection of magpies?'' 
    They shall rise up heroes, there will be many, 
    None will prevail against them at last. 
    They go saying each: ``I am one of many''; 
    Their hands empty save for history. 
    They die at bridges, bridge gates, and drawbridges. 
    Remember now there were others before; 
    The sepulchres are full at ford and bridgehead. 
    There will be children with flowers there, 
    And lambs and golden-eyed lions there, 
    And people remembering in the future. 

\hfill                                                                   [1936] 
\end{poem}



\begin{poem}
    {Requiem for the Spanish Dead}
    {Kenneth Rexroth}

    The great geometrical winter constellations 
    Lift up over the Sierra Nevada, 
    I walk under the stars, my feet on the known round earth. 
    My eyes following the lights of an airplane, 
    Red and green, growling deep into the Hyades. 
    The note of the engine rises, shrill, faint, 
    Finally inaudible, and the lights go out 
    In the southeast haze beneath the feet of Orion. \\

    As the sound departs I am chilled and grow sick 
    With the thought that has come over me. I see Spain 
    Under the black windy sky, the snow stirring faintly, 
    Glittering and moving over the pallid upland, 
    And men waiting, clutched with cold and huddled together, 
    As an unknown plane goes over them. It flies southeast 
    Into the haze above the lines of the enemy, 
    Sparks appear near the horizon under it. 
    After they have gone out the earth quivers 
    And the sound comes faintly. The men relax for a moment 
    And grow tense again as their own thoughts return to them. \\

    I see the unwritten books, the unrecorded experiments, 
    The unpainted pictures, the interrupted lives, 
    Lowered into the graves with the red flags over them. 
    I see the quick gray brains broken and clotted with blood, 
    Lowered each in its own darkness, useless in the earth. 
    Alone on a hilltop in San Francisco suddenly 
    I am caught in a nightmare, the dead flesh 
    Mounting over half the world presses against me. \\

    Then quietly at first and then rich and full-bodied, 
    I hear the voice of a young woman singing. 
    The emigrants on the corner are holding 
    A wake for their oldest child, a driverless truck 
    Broke away on the steep hill and killed him, 
    Voice after voice adds itself to the singing. 
    Orion moves westward across the meridian, 
    Rigel, Bellatrix, Betelgeuse, marching in order, 
    The great nebula glimmering in his loins. \\

\hfill                                                                               [1937] 
\end{poem}



\begin{poem}
    {On What Planet }
    {Kenneth Rexroth}

    Uniformly over the whole countryside 
    The warm air flows imperceptibly seaward; 
    The autumn haze drifts in deep bands 
    Over the pale water; 
    White egrets stand in the blue marshes; 
    Tamalpais, Diablo, St. Helena 
    Float in the air. 
    Climbing on the cliffs of Hunter's Hill 
    We look out over fifty miles of sinuous 
    Interpenetration of mountains and sea. \\

    Leading up a twisted chimney, 
    Just as my eyes rise to the level 
    Of a small cave, two white owls 
    Fly out, silent, close to my face. 
    They hover, confused in the sunlight, 
    And disappear into the recesses of the cliff. \\

    All day I have been watching a new climber, 
    A young girl with ash blond hair 
    And gentle confident eyes. 
    She climbs slowly, precisely, 
    With unwasted grace. 
    While I am coiling the ropes, 
    Watching the spectacular sunset, 
    She turns to me and says, quietly, 
    ``It must be very beautiful, the sunset, 
    On Saturn, with the rings and all the moons.'' 

\hfill                                                                                 [1937?/1940] 
\end{poem}



\begin{poem}
    {Climbing Milestone Mountain 
    August 22, 1937}
        {Kenneth Rexroth}

    For a month now, wandering over the Sierras, 
    A poem had been gathering in my mind, 
    Details of significance and rhythm, 
    The way poems do, but still lacking a focus. 
    Last night I remembered the date and it all 
    Began to grow together and take on purpose. 
    \hfill       We sat up late while Deneb moved over the zenith 
    And I told Marie all about Boston, how it looked 
    That last terrible week, how hundreds stood weeping 
    Impotent in the streets that last midnight. 
    I told her how those hours changed the lives of thousands, 
    How America was forever a different place 
    Afterwards for many. 
    \hfill                                In the morning 
    We swam in the cold transparent lake, the blue 
    Damsel flies on all the reeds like millions 
    Of narrow metallic flowers, and I thought 
    Of you behind the grille in Dedham, Vanzetti, 
    Saying, ``Who would ever have thought we would make this history?'' 
    Crossing the brilliant mile-square meadow 
    Illuminated with asters and cyclamen, 
    The pollen of the lodgepole pines drifting 
    With the shifting wind over it and the blue 
    And sulphur butterflies drifting with the wind, 
    I saw you in the sour prison light, saying, 
    ``Goodbye comrade.'' 
    \hfill                                      In the basin under the crest 
    Where the pines end and the Sierra primrose begins, 
    A party of lawyers was shooting at a whiskey bottle. 
    The bottle stayed on its rock, nobody could hit it. 
    Looking back over the peaks and canyons from the last lake, 
    The pattern of human beings seemed simpler 
    Than the diagonals of water and stone. 
    Climbing the chute, up the melting snow and broken rock, 
    I remembered what you said about Sacco, 
    How it slipped your mind and you demanded it be read into the record. 
    Traversing below the ragged ar\^ete, 
    One cheek pressed against the rock 
    The wind slapping the other, 
    I saw you both marching in an army 
    You with the red and black flag, Sacco with the rattlesnake banner. 
    I kicked steps up the last snow bank and came 
    To the indescribably blue and fragrant 
    Polemonium and the dead sky and the sterile 
    Crystalline granite and final monolith of the summit. 
    These are the things that will last a long time, Vanzetti, 
    I am glad that once on your day I have stood among them. 
    Some day mountains will be named after you and Sacco. 
    They will be here and your name with them, 
    ``When these days are but a dim remembering of the time 
    When man was wolf to man.'' 
    I think men will be remembering you a long time 
    Standing on the mountains 
    Many men, a long time, comrade. 

\hfill [1937] 
\end{poem}

\begin{poem}
{Proust's Madeleine}
{Kenneth Rexroth}

Somebody has given my
Baby daughter a box of
Old poker chips to play with.
Today she hands me one while
I am sitting with my tired
Brain at my desk. It is red.
On it is a picture of
An elk's head and the letters
B.P.O.E.?a chip from
A small town Elks' Club. I flip
It idly in the air and
Catch it and do a coin trick
To amuse my little girl.
Suddenly everything slips aside.
I see my father
Doing the very same thing,
Whistling ``Beautiful Dreamer,''
His breath smelling richly
Of whiskey and cigars. I can
Hear him coming home drunk
From the Elks' Club in Elkhart
Indiana, bumping the
Chairs in the dark. I can see
Him dying of cirrhosis
Of the liver and stomach
Ulcers and pneumonia,
Or, as he said on his deathbed, of
Crooked cards and straight whiskey,
Slow horses and fast women.
\end{poem}



% LocalWords:  murmurous aspens cottonwoods

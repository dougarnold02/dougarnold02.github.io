
\begin{poem}
{Dover Beach} 
{Matthew Arnold}

The sea is calm to-night.                                       
The tide is full, the moon lies fair				
Upon the straits; --- on the French coast the light		
Gleams and is gone; the cliffs of England stand,		
Glimmering and vast, out in the tranquil bay.			
Come to the window, sweet is the night-air!			
Only, from the long line of spray				
Where the sea meets the moon-blanch'd land,			
Listen! you hear the grating roar				
Of pebbles which the waves draw back, and fling,		
At their return, up the high strand,				
Begin, and cease, and then again begin,				
With tremulous cadence slow, and bring				
The eternal note of sadness in.\\

Sophocles long ago						
Heard it on the Aegean, and it brought				
Into his mind the turbid ebb and flow				
Of human misery; we						
Find also in the sound a thought,				
Hearing it by this distant northern sea.\\

The Sea of Faith						
Was once, too, at the full, and round earth's shore		
Lay like the folds of a bright girdle furl'd.			
But now I only hear						
Its melancholy, long, withdrawing roar,				
Retreating, to the breath					
Of the night-wind, down the vast edges drear			
And naked shingles of the world.\\
							
Ah, love, let us be true					
To one another! for the world, which seems			
To lie before us like a land of dreams,				
So various, so beautiful, so new,				
Hath really neither joy, nor love, nor light,			
Nor certitude, nor peace, nor help for pain;			
And we are here as on a darkling plain				
Swept with confused alarms of struggle and flight,		
Where ignorant armies clash by night.
\end{poem}

















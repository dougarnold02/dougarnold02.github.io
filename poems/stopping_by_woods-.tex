\begin{poem}
{After Apple Picking}
{Robert Frost}
     My long two-pointed ladder's sticking through a tree 
     Toward heaven still. 				  
     And there's a barrel that I didn't fill 		  
     Beside it, and there may be two or three 		  
     Apples I didn't pick upon some bough. 		  
     But I am done with apple-picking now. 		  
     Essence of winter sleep is on the night, 		  
     The scent of apples; I am drowsing off. 		  
     I cannot shake the shimmer from my sight 		  
     I got from looking through a pane of glass 	  
     I skimmed this morning from the water-trough, 	  
     And held against the world of hoary grass. 	  
     It melted, and I let it fall and break. 		  
     But I was well 					  
     Upon my way to sleep before it fell, 		  
     And I could tell 					  
     What form my dreaming was about to take. 		  
     Magnified apples appear and reappear, 		  
     Stem end and blossom end, 				  
     And every fleck of russet showing clear. 		  
     My instep arch not only keeps the ache, 		  
     It keeps the pressure of a ladder-round. 		  
     And I keep hearing from the cellar-bin 		  
     That rumbling sound 				  
     Of load on load of apples coming in. 		  
     For I have had too much 				  
     Of apple-picking; I am overtired 			  
     Of the great harvest I myself desired. 		  
     There were ten thousand thousand fruit to touch, 	  
     Cherish in hand, lift down, and not let fall, 	  
     For all 						  
     That struck the earth, 				  
     No matter if not bruised, or spiked with stubble, 	  
     Went surely to the cider-apple heap 		  
     As of no worth. 					  
     One can see what will trouble 			  
     This sleep of mine, whatever sleep it is. 		  
     Were he not gone, 					  
     The woodchuck could say whether it's like his 	  
     Long sleep, as I describe its coming on, 		  
     Or just some human sleep.                            
\end{poem}

\begin{poem}
{Birches}
{Robert Frost}

     When I see birches bend to left and right                    
     Across the lines of straighter darker trees, 		  
     I like to think some boy's been swinging them. 		  
     But swinging doesn't bend them down to stay. 		  
     Ice-storms do that. Often you must have seen them 		  
     Loaded with ice a sunny winter morning 			  
     After a rain. They click upon themselves 			  
     As the breeze rises, and turn many-colored 		  
     As the stir cracks and crazes their enamel. 		  
     Soon the sun's warmth makes them shed crystal shells 	  
     Shattering and avalanching on the snow-crust- 		  
     Such heaps of broken glass to sweep away 			  
     You'd think the inner dome of heaven had fallen. 		  
     They are dragged to the withered bracken by the load, 	  
     And they seem not to break; though once they are bowed 	  
     So low for long, they never right themselves: 		  
     You may see their trunks arching in the woods 		  
     Years afterwards, trailing their leaves on the ground 	  
     Like girls on hands and knees that throw their hair 	  
     Before them over their heads to dry in the sun. 		  
     But I was going to say when Truth broke in 		  
     With all her matter-of-fact about the ice-storm 		  
     (Now am I free to be poetical?) 				  
     I should prefer to have some boy bend them 		  
     As he went out and in to fetch the cows- 			  
     Some boy too far from town to learn baseball, 		  
     Whose only play was what he found himself, 		  
     Summer or winter, and could play alone. 			  
     One by one he subdued his father's trees 			  
     By riding them down over and over again 			  
     Until he took the stiffness out of them, 			  
     And not one but hung limp, not one was left 		  
     For him to conquer. He learned all there was 		  
     To learn about not launching out too soon 			  
     And so not carrying the tree away 				  
     Clear to the ground. He always kept his poise 		  
     To the top branches, climbing carefully 			  
     With the same pains you use to fill a cup 			  
     Up to the brim, and even above the brim. 			  
     Then he flung outward, feet first, with a swish, 		  
     Kicking his way down through the air to the ground. 	  
     So was I once myself a swinger of birches. 		  
     And so I dream of going back to be. 			  
     It's when I'm weary of considerations, 			  
     And life is too much like a pathless wood 			  
     Where your face burns and tickles with the cobwebs 	  
     Broken across it, and one eye is weeping 			  
     From a twig's having lashed across it open. 		  
     I'd like to get away from earth awhile 			  
     And then come back to it and begin over. 			  
     May no fate willfully misunderstand me 			  
     And half grant what I wish and snatch me away 		  
     Not to return. Earth's the right place for love: 		  
     I don't know where it's likely to go better. 		  
     I'd like to go by climbing a birch tree, 			  
     And climb black branches up a snow-white trunk 		  
     Toward heaven, till the tree could bear no more, 		  
     But dipped its top and set me down again. 			  
     That would be good both going and coming back. 		  
     One could do worse than be a swinger of birches.             
\end{poem}

\begin{poem}
{Mending Wall }
{Robert Frost}
     Something there is that doesn't love a wall,                
     That sends the frozen-ground-swell under it, 		 
     And spills the upper boulders in the sun, 			 
     And makes gaps even two can pass abreast. 			 
     The work of hunters is another thing: 			 
     I have come after them and made repair 			 
     Where they have left not one stone on a stone, 		 
     But they would have the rabbit out of hiding, 		 
     To please the yelping dogs. The gaps I mean, 		 
     No one has seen them made or heard them made, 		 
     But at spring mending-time we find them there. 		 
     I let my neighbort know beyond the hill; 			 
     And on a day we meet to walk the line 			 
     And set the wall between us once again. 			 
     We keep the wall between us as we go. 			 
     To each the boulders that have fallen to each. 		 
     And some are loaves and some so nearly balls 		 
     We have to use a spell to make them balance: 		 
     `Stay where you are until our backs are turned!' 		 
     We wear our fingers rough with handling them. 		 
     Oh, just another kind of out-door game, 			 
     One on a side. It comes to little more: 			 
     There where it is we do not need the wall: 		 
     He is all pine and I am apple orchard. 			 
     My apple trees will never get across 			 
     And eat the cones under his pines, I tell him. 		 
     He only says, `Good fences make good neighhours'. 		 
     Spring is the mischief in me, and I wonder 		 
     If I could put a notion in his head: 			 
     `Why do they make good neighbours? Isn't it 		 
     Where there are cows? 					 
     But here there are no cows. 				 
     Before I built a wall I'd ask to know 			 
     What I was walling in or walling out, 			 
     And to whom I was like to give offence. 			 
     Something there is that doesn't love a wall, 		 
     That wants it down.' I could say `.Elves' to him, 		 
     But it's not elves exactly, and I'd rather 		 
     He said it for himself. I see him there 			 
     Bringing a stone grasped firmly by the top 		 
     In each hand, like an old-stone savage armed. 		 
     He moves in darkness as it seems to me ----		 
     Not of woods only and the shade of trees. 			 
     He will not go behind his father's saying, 		 
     And he likes having thought of it so well 			 
     He says again, Good fences make good neighbours.            
\end{poem}



\begin{poem}
{On Looking Up by Chance at the Constellations}
{Robert Frost}
     You'll wait a long, long time for anything much                 
     To happen in heaven beyond the floats of cloud 		     
     And the Northern Lights that run like tingling nerves. 	     
     The sun and moon get crossed, but they never touch, 	     
     Nor strike out fire from each other nor crash out loud. 	     
     The planets seem to interfere in their curves --- 		     
     But nothing ever happens, no harm is done. 		     
     We may as well go patiently on with our life, 		     
     And look elsewhere than to stars and moon and sun 		     
     For the shocks and changes we need to keep us sane. 	     
     It is true the longest drout will end in rain, 		     
     The longest peace in China will end in strife. 		     
     Still it wouldn't reward the watcher to stay awake 	     
     In hopes of seeing the calm of heaven break 		     
     On his particular time and personal sight. 		     
     That calm seems certainly safe to last to-night.                
\end{poem}

\begin{poem}
{Stopping in the Woods on a Snowy Evening}
{Robert Frost}

     Whose woods these are I think I know  
     His house is in the village though;   
     He will not see me stopping here 	   
     To watch his woods fill up with snow.\\

     My little horse must think it queer 
     To stop without a farmhouse near 	 
     Between the woods and frozen lake 	 
     The darkest evening of the year.    \\

     He gives his harness bells a shake   
     To ask if there is some mistake. 	  
     The only other sound's the sweep 	  
     Of easy wind and downy flake.       \\

     The woods are lovely, dark and deep. 
     But I have promises to keep, 	  
     And miles to go before I sleep. 	  
     And miles to go before I sleep.     
\end{poem}


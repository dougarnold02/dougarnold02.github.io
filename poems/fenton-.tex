\begin{poem}
{In Paris with You}
{James Fenton}

 Don't talk to me of love.  I've had an earful
 And I get tearful when I've downed a drink or two.
 I'm one of your talking wounded.
 I'm a hostage. I'm maroonded.
 But I'm in Paris with you.\\
 
 Yes, I'm angry at the way I've been bamboozled
 And resentful at the mess that I've been through.
 I admit I'm on the rebound
 And I don't care where are we bound.
 I'm in Paris with you.\\
 
 Do you mind if we do not go to the Louvre,
 If we say sod off to sodding Notre Dame
 If we skip the champs Elysees
 And remain here in this sleazy
 Old hotel room
 Doing this or that
 To what and whom
 Learning who you are,
 Learning what I am.\\
 
 Don't talk to me of love. Let's talk of Paris,
 The little bit of Paris in our view.
 There's that crack across the ceiling
 And the hotel walls are peeling 
 And I'm in Paris with you.\\
 
 Don't talk to me of love.  Let's talk of Paris.
 I'm in Paris with the slightest thing you do.
 I'm in Paris with your eyes, your mouth,
 I'm in Paris with \ldots all points south.
 Am I embarrassing you?
 I'm in Paris with you.\\
 \end{poem}
 \begin{poem}
{Out of the East}
{James Fenton}

 Out of the South came Famine.
 Out of the West came Strife.
 Out of the North came a storm cone
 And out of the East came a warrior wind
 And it struck you like a knife.
 Out of the East there shone a sun
 As the blood rose on the day
 And it shone on the work of the warrior wind
 And it shone on the heart
 And it shone on the soul
 And they called the sun --- Dismay.\\
 
 And it's a far cry from the jungle
 To the city of Phnom Penh
 And many try
 And many die
 Before they can see their homes again
 And it's a far cry from the paddy track
 To the palace of the king
 And many go
 Before they know
 It's a far cry.
 It's a war cry.
 Cry for the war that can do this thing.\\
 
 A foreign soldier came to me
 And he gave me a gun
 And he predicted victory
 Before the year was done.\\
 
 He taught me how to kill a man.
 He taught me how to try.
 Be he forgot to say to me
 How an honest man should die.\\
 
 He taught me how to kill a man
 Who was my enemy
 But never how to kill a man
 Who'd been a friend to me.\\
 
 You fought the way a hero fights ---
 You had no need to fear
 My friend, but you are wounded now
 And I'm not allowed to leave you here\\
 
 Alive.\\
 
 Out of the East came Anger
 And it walked a dusty road
 And it stopped when it came to a river bank
 And it pitched a camp
 And it gazed across
 To where the city stood
 When
 Out of the West came thunder
 But it came without a sound
 For it came at the speed of the warrior wind
 And it fell on the heart
 And it fell on the soul
 And it shook the battleground\\
 
 And it's a far cry from the cockpit
 To the foxhole in the clay
 And we were a
 Coordinate
 In a foreign land
 Far away
 And it's a far cry from the paddy track
 To the palace of the king
 And many try
 And they ask why
 It's a far cry.
 It's a war cry.
 Cry for the war that can do this thing.\\
 
 Next year the army came for me
 And I was sick and thin
 And they put a weapon in our hands
 And they told us we would win\\
 
 And they feasted us for seven days
 And they slaughtered a hundred cattle
 And we sang our songs of victory
 And the glory of the battle\\
 
 And they sent us down the dusty roads
 In the stillness of the night
 And when the city heard from us
 It burst in a flower of light.\\
 
 The tracer bullets found us out.
 The guns were never wrong
 And the gunship said Regret Regret
 The words of your victory song.\\
 
 Out of the North came an army
 And it was clad in black
 And out of the South came a gun crew
 With a hundred shells
 And a howitzer
 And we walked in black along the paddy track
 When
 Out of the West came napalm
 And it tumbled from the blue
 And it spread at the speed of the warrior wind
 And it clung to the heart
 And it clung to the soul
 As napalm is designed to do\\
 
 And it's a far cry from the fireside
 To the fire that finds you there
 In the foxhole
 By the temple gate
 The fire that finds you everywhere
 And it's a far cry from the paddy track
 To the palace of the king
 And many try
 And they ask why
 It's a far cry.
 It's a war cry.
 Cry for the war that can do this thing.\\
 
 My third year in the army
 I was sixteen years old
 And I had learnt enough, my friend,
 To believe what I was told\\
 
 And I was told that we would take
 The city of Phnom Penh
 And they slaughtered all the cows we had
 And they feasted us again\\
 
 And at last we were given river mines
 And we blocked the great Mekong
 And now we trained our rockets on
 The landing-strip at Pochentong.\\
 
 The city lay within our grasp.
 We only had to wait.
 We only had to hold the line
 By the foxhole, by the temple gate\\
 
 When
 Out of the West came clusterbombs
 And they burst in a hundred shards
 And every shard was a new bomb
 And it burst again
 Upon our men
 As they gasped for breath in the temple yard.
 Out of the West came a new bomb
 And it sucked away the air
 And it sucked at the heart
 And it sucked at the soul
 And it found a lot of children there\\
 
 And it's a far cry from the temple yard
 To the map of the general staff
  From the grease pen to the gasping men
 To the wind that blows the soul like chaff
 And it's a far cry from the paddy track
 To the palace of the king
 And many go
 Before they know
 It's a far cry.
 It's a war cry.
 Cry for the war that has done this thing.\\
 
 A foreign soldier came to me
 And he gave me a gun
 And the liar spoke of victory
 Before the year was done.\\
 
 What would I want with victory
 In the city of Phnom Penh?
 Punish the city! Punish the people!
 What would I want but punishment?\\
 
 We have brought the king home to his palace.
 We shall leave him there to weep
 And we'll go back along the paddy track
 For we have promises to keep.\\
 
 For the promise made in the foxhole,
 For the oath in the temple yard,
 For the friend I killed on the battlefield
 I shall make that punishment hard.\\
 
 Out of the South came Famine.
 Out of the West came Strife.
 Out of the North came a storm cone
 And out of the East came a warrior wind
 And it struck you like a knife.
 Out of the East there shone a sun
 As the blood rose on the day
 And it shone on the work of the warrior wind
 And it shone on the heart
 And it shone on the soul
 And they called the sun Dismay, my friend,
 They called the sun --- Dismay.\\

\end{poem}

\begin{poem}
{The Ballad of the Imam and the Shah}
{James Fenton}
(An Old Persian Legend)
to C. E. H.\\

 It started with a stabbing at a well
 Below the minarets of Isfahan.
 The widow took her son to see them kill
 The officer who'd murdered her old man.
 The child looked up and saw the hangman's work ---
 The man who'd killed his father swinging high,
 The mother said: 'My child, now be at peace.
 The wolf has had the fruits of all his crime.'\\

\textit{
  From felony to felony to crime
  From robbery to robbery to loss
  From calumny to calumny to spite
  From rivalry to rivalry to zeal}\\

 All this was many centuries ago ---
 The kind of thing that couldn't happen now ---
 When Persia was the empire of the Shah
 And many were the furrows on his brow.
 The peacock the symbol of his throne
 And many were the jewels and its eyes
 And many were the prisons in the land
 And many were the torturers and spies.\\

\textit{
  From tyranny to tyranny to war
  From dynasty to dynasty to hate
  From villainy to villainy to death
  From policy to policy to grave}\\

 The child grew up a clever sort of chap
 And he became a mullah, like his dad ---
 Spent many years in exile and disgrace
 Because he told the world the Shah was bad.
 'Believe in God,' he said, 'believe in me.
 Believe me when I tell you who I am.
 Now chop the arm of wickedness away.
 Hear what I say, I am the great Imam.'\\

\textit{
  From heresy to heresy to fire
  From clerisy to clerisy to fear
  From litany to litany to sword
  From fallacy to fallacy to wrong}\\

 And so the Shah was forced to flee abroad.
 The Imam was the ruler in his place.
 He started killing everyone he could
 To make up for the years of his discgrace.
 And when there were no enemies at home
 He sent his men to Babylon to fight.
 And when he'd lost an army in that way
 He knew what God was telling him was right.\\

\textit{
  From poverty to poverty to wrath
  From agony to agony to doubt
  From malady to malady to shame
  From misery to misery to fight}\\

 He sent the little children out to war.
 They went out with his portrait in their hands.
 The desert and the marshes filled with blood.
 The mothers heard the news in Isfahan.
 Now Babylon is buried under dirt.
 Persepolis is peeping through the sand.
 The child who saw his father's killer killed
 Has slaughtered half the children in the land.\\

\textit{
 From felony
 to robbery
 to calumny
 to rivalry
 to tyranny
 to dynasty
 to villainy
 to policy
 to heresy
 to clerisy
 to litany
 to fallacy
 to poverty
 to agony
 to malady
 to misery ---}\\

 The song is yours. Arrange it as you will.
 Remember where each word fits in the line
 And every combination will be true
 And every permutation will be fine:\\

\textit{
  From policy to felony to fear
  From litany to heresy to fire
  From villainy to tyranny to war
  From tyranny to dynasty to shame}\\

\textit{
  From poverty to malady to grave
  From malady to agony to spite
  From agony to misery to hate
  From misery to policy to fight!}\\
 \end{poem}

 \begin{poem}
{Out of Danger}
{James Fenton}
 Heart be kind and sign the release
 As the trees their loss approve.
 Learn as leaves must learn to fall
 Out of danger, out of love.\\
   
 What belongs to frost and thaw
 Sullen winter will not harm.
 What belongs to wind and rain
 Is out of danger from the storm.\\
   
 Jealous passion, cruel need
 Betray the heart they feed upon.
 But what belongs to earth and death
 Is out of danger from the sun.\\
   
 I was cruel, I was wrong ---
 Hard to say and hard to know.
 You do not belong to me.
 You are out of danger now ---\\
   
 Out of danger from the wind,
 Out of danger from the wave,
 Out of danger from the heart
 Falling, falling out of love.\\
 
\end{poem}
\begin{poem}
  {For Andrew Wood}
  {James Fenton}
What would the dead want from us
Watching from their cave?
Would they have us forever howling?
Would they have us rave
Or disfigure ourselves, or be strangled
Like some ancient emperor’s slave?\\

None of my dead friends were emperors
With such exorbitant tastes
And none of them were so vengeful
As to have all their friends waste
Waste quiet away in sorrow
Disfigured and defaced.\\

I think the dead would want us
To weep for what they have lost.
I think that our luck in continuing
Is what would affect them most.
But time would find them generous
And less self-engrossed.\\

And time would find them generous
As they used to be
And what else would they want from us
But an honored place in our memory,
A favorite room, a hallowed chair,
Privilege and celebrity?
And so the dead might cease to grieve
And we might make amends
And there might be a pact between
Dead friends and living friends.
What our dead friends would want from us
Would be such living friends.\\
\end{poem}
%% Local Variables:
%% TeX-master: "poems.tex"
%% End:

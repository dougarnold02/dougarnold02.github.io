\begin{poem}
{Snake}
{D H Lawrence}
 
 A snake came to my water-trough
 On a hot, hot day, and I in pyjamas for the heat,
 To drink there.\\
 
 In the deep, strange-scented shade of the great dark carob tree
 I came down the steps with my pitcher
 And must wait, must stand and wait, for there he was at the trough before me.\\
 
 He reached down from a fissure in the earth-wall in the gloom
 And trailed his yellow-brown slackness soft-bellied down, over the 
   edge of the stone trough
 And rested his throat upon the stone bottom,
 And where the water had dripped from the tap, in a small clearness,
 He sipped with his straight mouth,
 Softly drank through his straight gums, into his slack long body,
 Silently.\\
 
 Someone was before me at my water-trough,
 And I, like a second-comer, waiting.\\
 
 He lifted his head from his drinking, as cattle do,
 And looked at me vaguely, as drinking cattle do,
 And flickered his two-forked tongue from his lips, and mused a  moment,
 And stooped and drank a little more,
 Being earth-brown, earth-golden from the burning bowels of the earth
 On the day of Sicilian July, with Etna smoking.\\
 
 The voice of my education said to me
 He must be killed,
 For in Sicily the black, black snakes are innocent, the gold are venomous.
 And voices in me said, If you were a man
 You would take a stick and break him now, and finish him off.\\
 
 But must I confess how I liked him,
 How glad I was he had come like a guest in quiet, to drink at my water-trough
 And depart peaceful, pacified, and thankless,
 Into the burning bowels of this earth?\\
 
 Was it cowardice, that I dared not kill him?
 Was it perversity, that I longed to talk to him?
 Was it humility, to feel so honoured?
 I felt so honoured.\\
 
 And yet those voices:
 If you were not afraid, you would kill him!\\
 
 And truly I was afraid, I was most afraid,
 But even so, honoured still more
 That he should seek my hospitality
 From out the dark door of the secret earth.\\
 
 He drank enough
 And lifted his head, dreamily, as one who has drunken,
 And flickered his tongue like a forked night on the air, so black,
 Seeming to lick his lips,
 And looked around like a god, unseeing, into the air,
 And slowly turned his head,
 And slowly, very slowly, as if thrice adream,
 Proceeded to draw his slow length curving round
 And climb again the broken bank of my wall-face.\\
 
 And as he put his head into that dreadful hole,
 And as he slowly drew up, snake-easing his shoulders, and entered farther,
 A sort of horror, a sort of protest against his withdrawing into 
   that horrid black hole,
 Deliberately going into the blackness, and slowly drawing himself after,
 Overcame me now his back was turned.\\
 
 I looked round, I put down my pitcher,
 I picked up a clumsy log
 And threw it at the water-trough with a clatter.\\
 
 I think it did not hit him,
 But suddenly that part of him that was left behind convulsed in 
   undignified haste,
 Writhed like lightning, and was gone
 Into the black hole, the earth-lipped fissure in the wall-front,
 At which, in the intense still noon, I stared with fascination.\\
 
 And immediately I regretted it.
 I thought how paltry, how vulgar, what a mean act!
 I despised myself and the voices of my accursed human education.\\
 
 And I thought of the albatross,
 And I wished he would come back, my snake.\\
 
 For he seemed to me again like a king,
 Like a king in exile, uncrowned in the underworld,
 Now due to be crowned again.\\
 
 And so, I missed my chance with one of the lords
 Of life.
 And I have something to expiate:
 A pettiness.\\
 
\end{poem}

\begin{poem}
{Menelaus and Helen}
{Rupert Brooke}
\newcommand{\Ind}{\Indent[2em]}
\Stanza{I}


Hot through Troy's ruin Menelaus broke
\Ind   To Priam's palace, sword in hand, to sate
\Ind   On that adulterous whore a ten years' hate
And a king's honour. Through red death, and smoke,
And cries, and then by quieter ways he strode,
\Ind   Till the still innermost chamber fronted him.
\Ind   He swung his sword, and crashed into the dim
Luxurious bower, flaming like a god. \\

High sat white Helen, lonely and serene.
\Ind   He had not remembered that she was so fair,
And that her neck curved down in such a way;
And he felt tired. He flung the sword away,
\Ind   And kissed her feet, and knelt before her there,
The perfect Knight before the perfect Queen. \\


\Stanza{II}

So far the poet. How should he behold
\Ind   That journey home, the long connubial years?
\Ind   He does not tell you how white Helen bears
Child on legitimate child, becomes a scold,
Haggard with virtue. Menelaus bold
\Ind   Waxed garrulous, and sacked a hundred Troys
\Ind   'Twixt noon and supper. And her golden voice
Got shrill as he grew deafer. And both were old. \\

Often he wonders why on earth he went
\Ind   Troyward, or why poor Paris ever came.
Oft she weeps, gummy-eyed and impotent;
\Ind   Her dry shanks twitch at Paris' mumbled name.
So Menelaus nagged; and Helen cried;
And Paris slept on by Scamander side. 
\end{poem}

\begin{poem}
{Adam's Complaint}
{Denise Levertov}
 Some people,
 no matter what you give them,
 still want the moon.\\

 The bread,
 the salt,
 white meat and dark,
 still hungry.\\

 The marriage bed
 and the cradle,
 still empty arms.\\

 You give them land,
 their own earth under their feet,
 still they take to the roads.\\

 And water: dig them the deepest well,
 still it's not deep enough
 to drink the moon from.
\end{poem}



\begin{poem}
{Named}
{Stephen Dunn}
He'd spent his life trying to control the names people gave him;
oh the unfair and the accurate equally hurt.\\

Just recently he'd been a son-of-a-bitch and sweetheart in the same day,
and once again knew what antonyms\\

love and control are, and how comforting it must be to have a business card ---
Manager, Specialist --- and believe what it says.\\

Who, in fact, didn't want his most useful name to enter with him,
when he entered a room, who didn't want to be\\

that kind of lie? A man who was a sweetheart and a son-of-a-bitch
was also more or less every name\\

he'd ever been called, and when you die, he thought, that's when it happens,
you're collected forever into a few small words.\\

But never to have been outrageous or exquisite, no grand mistake
so utterly yours it causes whispers\\

in the peripheries of your presence --- that was his fear.
``Reckless''; he wouldn't object to such a name\\

if it came from the right voice with the right amount of reverence.
Someone nearby, of course, certain to add ``fool.'' 
\end{poem}


\begin{poem}
{The Leaden-Eyed}
{Vachel Lindsay}
 Let not young souls be smothered out before
 They do quaint deeds and fully flaunt their pride.
 It is the world's one crime its babes grow dull,
 Its poor are ox-like, limp and leaden-eyed.\\

 Not that they starve, but starve so dreamlessly;
 Not that they sow, but that they seldom reap;
 Not that they serve, but have no gods to serve;
 Not that they die, but that they die like sheep.
\end{poem}

\begin{poem}
{The More you have to loose}
{David Lehman}

Time lies, and a year can go by in a day
Look at your watch. Do your eyes say 2.45 or 9.15?
The more you have, the more you can give away.\\

Your know the feeling, having no money, having to stay
With relatives when you travel, unable to say what you mean:
Time lies, and a year can go by in a day.\\

When my father turned into my son, like a play,
All the fun took place off-stage. What about the missing queen?
The more you have, the more you can give away.\\

The less you believe, the more you wish you could pray.
Like a clock without hands, the truth of a face remains unseen.
Time lies, and a year can go by in a day.\\

With an elbow on the counter, and no passions left to sway,
The all-night waitress smokes butt after butt, coughing in-between: 
The more you have, the more you can throw away.\\

Ocean, what is on the other side of all that blue and grey?
What does the grass know of yesterday's vanished green?
Time lies, and a year can go by in a day.
The more you have, the more you can give away.
\end{poem}


\begin{poem}
{Je souhaite dans ma maison}
{Apollinaire}
Je souhaite dans ma maison
Une jeunne fille ayant sa raison
Un chat passant parmi les livres
Des amis en tout saison
Sans lesquels je ne peux pas vivre
\end{poem}

\begin{poem}
 {Black Rook in Rainy Weather}
 {Sylvia Plath}

 On the stiff twig up there 
 Hunches a wet black rook
 Arranging and rearranging its feathers in the rain-
 I do not expect a miracle
 Or an accident\\

 To set the sight on fire
 In my eye, nor seek
 Any more in the desultory weather some design,
 But let spotted leaves fall as they fall
 Without ceremony, or portent.\\

 Although, I admit, I desire,
 Occasionally, some backtalk
 From the mute sky, I can't honestly complain:
 A certain minor light may still 
 Lean incandescent\\

 Out of kitchen table or chair
 As if a celestial burning took
 Possession of the most obtuse objects now and then --
 Thus hallowing an interval
 Otherwise inconsequent \\

 By bestowing largesse, honor
 One might say love. At any rate, I now walk
 Wary (for it could happen
 Even in this dull, ruinous landscape); sceptical
 Yet politic, ignorant\\

 Of whatever angel any choose to flare
 Suddenly at my elbow. I only know that a rook
 Ordering its black feathers can so shine
 As to seize my senses, haul 
 My eyelids up, and grant \\

 A brief respite from fear
 Of total neutrality. With luck,
 Trekking stubborn through this season
 Of fatigue, I shall 
 Patch together a content\\

 Of sorts. Miracles occur.
 If you care to call those spasmodic
 Tricks of radiance
 Miracles. The wait's begun again,
 The long wait for the angel,\\

 For that rare, random descent.
\end{poem}


\begin{poem}
{Portrait of a Schoolmaster}
{Peter Levi}

And I should wish to draw you, caught so
head on one side, hand on the tea-table
that nervous posture and those eyes able
to relax in accuracy at a window,\\

in a view of hills reclined distantly,
olive coloured and tall, or like clear
Latin speech, weighed and fine in the ear,
designed by ice, or tortuous irony;\\

or else exhaling aphorisms like brittle
fire or a nineteenth century rocket,
--- by fascination seeming to forget
what this lights up so briefly and so little:\\

the mildewed landscape which no mind can cure,
hypocrisy of the hearts incompetence,
all the sad images of violence
and decorous religions of failure.\\

(This poem may be about Frederick Turner (1910-2000), a teacher at Stoneyhurst College.)
\end{poem}


\begin{poem}
{Concerto for Double Bass}
{John Fuller}

 He is a drunk leaning companionably
 Around a lamp post or doing up
 With intermittent concentration
 Another drunk's coat.\\

 He is a polite but devoted Valentino,
 Cheek to cheek, forgetting the next step.
 He is feeling the pulse of the fat lady
 Or cutting her in half.\\
 
 But close your eyes and it is sunset
 At the edge of the world. It is the language
 Of dolphins, the growth of tree-roots,
 The heart-beat slowing down.\\

\end{poem}

\begin{poem}
{Prayer}
{Carol Ann Duffy}
 
 Some days, although we cannot pray, a prayer
 utters itself. So, a woman will lift
 her head from the sieve of her hands and stare
 at the minims sung by a tree, a sudden gift.\\
 
 Some nights, although we are faithless, the truth
 enters our hearts, that small familiar pain;
 then a man will stand stock-still, hearing his youth
 in the distant Latin chanting of a train.\\
 
 Pray for us now. Grade 1 piano scales
 console the lodger looking out across
 a Midlands town. Then dusk, and someone calls
 a child's name as though they named their loss.\\
 
 Darkness outside. Inside, the radio's prayer ---
 Rockall. Malin. Dogger. Finisterre.\\
\end{poem}
 

\begin{poem}
{Epitaph for Francis Chartres (1669-1731)}
{Dr. Arbuthnot\\
(The London Magazine, 1732)}

\textsc{Here} continueth to rot
 The body of \textsc{Francis} \textsc{Chartres}
 Who with an \textsc{inflexible constancy}
\Indent and \textsc{inimitable uniformity} of Life  \textsc{persisted}
 In spite of \textsc{age} and \textsc{infirmities},
 In the practice of \textsc{every human vice};
 Excepting \textsc{prodigiality} and \textsc{hypocrisy}:
 His insatiable \textsc{avarice} exempted him from the first,
 His matchless \textsc{impudence} from the second.\\
\Indent Nor was he more singular
 In the undeviating \emph{pravity} of his \emph{Manners},
 Than successful
 In \emph{accumulating} wealth,
 For without \textsc{trade} or \textsc{profession},
 Without \textsc{trust} of \textsc{public money},
 And without \textsc{BRIBE-worthy} Service,
 He acquired, or more properly created,
 A \textsc{ministerial estate}.\\
\Indent He was the only Person of his Time,
 Who could cheat without the Mask of \textsc{honesty},
 Retain his Primeval \textsc{meanness}
 When possess'd of \textsc{ten thousand} a Year,
 And having daily deserved the Gibbet for what he \emph{did},
 Was at last condemned to it for what he \emph{could} not \emph{do}.\\
\Indent Oh Indignant Reader!
 Think not his life useless to Mankind!
 Providence conniv'd at his execrable Designs,
 To give to After-ages
 A conspicuous \textsc{proof} and \textsc{example},
 Of how small Estimation is \textsc{exorbitant wealth} in the sight of God,
 By bestowing it on the most \textsc{unworthy} of \textsc{all mortals}.
\end{poem}

\begin{poem}
  {Untitled}
  {Issa}
  What good luck!
  Bitten by
  This year's mosquitoes too.
\end{poem}

\begin{poem}
   {Adlestrop}
   {Edward Thomas}
 Yes, I remember Adlestrop --
 The name, because one afternoon
 Of heat the express-train drew up there
 Unwontedly. It was late June. \\

 The steam hissed. Someone cleared his throat.
 No one left and no one came
 On the bare platform. What I saw
 Was Adlestrop -- only the name \\

 And willows, willow-herb, and grass,
 And meadowsweet, and haycocks dry,
 No whit less still and lonely fair
 Than the high cloudlets in the sky.\\ 

 And for that minute a blackbird sang
 Close by, and round him, mistier,
 Farther and farther, all the birds
 Of Oxfordshire and Gloucestershire.
\end{poem}

\begin{poem}
  {Bearhug}
  {Michael Ondaatje}
 Griffin calls to come and kiss him goodnight
 I yell ok. Finish something I'm doing,
 then something else, walk slowly round
 the corner to my son's room.
 He is standing arms outstretched
 waiting for a bearhug. Grinning.\\
 
 Why do I give my emotion an animal's name,
 give it that dark squeeze of death?
 This is the hug which collects
 all his small bones and his warm neck against me.
 The thin tough body under the pyjamas
 locks to me like a magnet of blood.\\
 
 How long was he standing there
 like that, before I came?
\end{poem}

\begin{poem}
   {Strugnell's Sonnets (VI)}
   {Wendy Cope}

 Let me not to the marriage of true swine
 Admit impediments. With his big car
 He's won your heart, and you have punctured mine. 
 I have no spare; henceforth I'll bear the scar.
 Since women are not worth the booze you buy them
 I dedicate myself to Higher Things. 
 If men deride and sneer, I shall defy them
 And soar above Tulse Hill on poet's wings -- 
 A brother to the thrush in Brockwell Park,
 Whose song, though sometimes drowned by rock guitars,
 Outlives their din. One day I'll make my mark,
 Although I'm not from Ulster or from Mars,
 And when I'm published in some classy mag
 You'll rue the day you scarpered in his Jag.
\end{poem}

\begin{poem}
{Keith Chegwin as Fleance}
{Paul Farley}
The next rung up from extra and dogsbody
and all the cliches are true --- days waiting for 
enough light, learning card games, penny-ante,
 while fog rolls off the sea, a camera 
gets moisture in its gate, and Roman Polanski 
curses the day he chose Snowdonia.\\

He picked you for your hair to play this role:
 a look had reached Bootle from Altamont 
that year. You wouldn't say you sold your soul 
but learned your line inside a beating tent 
by candlelight, the shingle dark as coal
 behind each wave, and its slight restatement.\\

``A tale told by an idiot\ldots'' ``Not your turn, 
but perhaps, with time and practice\ldots'', the Pole starts. 
Who's to say, behind the accent and that grin, 
what designs you had on playing a greater part? 
The crew get ready while the stars go in. 
You speak the words you'd written on your heart\\

just as the long-awaited sunrise fires 
the sky a blueish pink. Who could have seen 
this future in the late schedules, where I 
can't sleep, and watch your flight from the big screen; 
on the other side of drink and wondering why,
the zany, household-name years in between?
\end{poem}

\begin{poem}
  {They Flee From Me}
  {Thomas Wyatt}

They flee from me that sometime did me seek
With naked foot stalking in my chamber.
I have seen them gentle tame and meek
That now are wild and do not remember
That sometime they put themselves in danger
To take bread at my hand; and now they range
Busily seeking with a continual change.

Thanked be fortune, it hath been otherwise
Twenty times better; but once in special,
In thin array after a pleasant guise,
When her loose gown from her shoulders did fall,
And she me caught in her arms long and small;
And therewithal sweetly did me kiss,
And softly said, Dear heart, how like you this?

It was no dream, I lay broad waking.
But all is turned thorough my gentleness
Into a strange fashion of forsaking;
And I have leave to go of her goodness
And she also to use newfangleness.
But since that I so kindely am served,
I would fain know what she hath deserved.
\end{poem}

\begin{poem}
{The South Country}
{Hilaire Belloc}

When I am living in the Midlands
That are sodden and unkind,
I light my lamp in the evening:
My work is left behind;
And the great hills of the South Country
Come back into my mind.\\

The great hills of the South Country
They stand along the sea;
And it's there walking in the high woods
That I could wish to be,
And the men that were boys when I was a boy
Walking along with me.\\

The men that live in North England
I saw them for a day:
Their hearts are set upon the waste fells,
Their skies are fast and grey;
From their castle-walls a man may see
The mountains far away.\\

The men that live in West England
They see the Severn strong,
A-rolling on rough water brown
Light aspen leaves along.
They have the secret of the Rocks,
And the oldest kind of song.\\

But the men that live in the South Country
Are the kindest and most wise,
They get their laughter from the loud surf,
And the faith in their happy eyes
Comes surely from our Sister the Spring
When over the sea she flies;
The violets suddenly bloom at her feet,
She blesses us with surprise.\\

I never get between the pines
But I smell the Sussex air;
Nor I never come on a belt of sand
But my home is there.
And along the sky the line of the Downs
So noble and so bare.\\

A lost thing could I never find,
Nor a broken thing mend:
And I fear I shall be all alone
When I get towards the end.
Who will there be to comfort me
Or who will be my friend?\\

I will gather and carefully make my friends
Of the men of the Sussex Weald;
They watch the stars from silent folds,
They stiffly plough the field.
By them and the God of the South Country
My poor soul shall be healed.\\

If I ever become a rich man,
Or if ever I grow to be old,
I will build a house with deep thatch
To shelter me from the cold,
And there shall the Sussex songs be sung
And the story of Sussex told.\\

I will hold my house in the high wood
Within a walk of the sea,
And the men that were boys when I was a boy
Shall sit and drink with me.\\
\end{poem}

% LocalWords:  mildewed coloured continueth infirmities prodigiality pravity
% LocalWords:  possess'd conniv'd {Je maison} raison maison bestowing
% LocalWords:  Une jeunne fille ayant sa raison Un chat passant parmi les livres
% LocalWords:  Des amis en tout saison Sans lesquels je ne peux pas vivre
